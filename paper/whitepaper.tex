% NOTE: It may be necessary to run LaTeX two or more times
% in order for section headings, internal and external
% references, etc. to display properly

%————————————————————————————
% DRAFT MODE
%The following is for drafts and produces
% a single column of double-spaced text
% with a wide right margin for editor's
% notes

\documentclass[preprint,pre,floats,aps,amsmath,amssymb]{revtex4-1}
\usepackage{geometry}
  \geometry{letterpaper,right=2in}
%————————————————————————————
% JOURNAL MODE
% The following line is for publication and produces
% two columns of single-spaced text
%\documentclass[twocolumn,pre,floats,aps,amsmath,amssymb]{revtex4}

%————————————————————————————
% PACKAGES USED IN THIS DOCUMENT
\usepackage{graphicx}
\usepackage{bm}

%————————————————————————————

\begin{document}

%\title{DuraChain: A Decentralized Electronic Healthcare Event Environment}
%\author{Brandon JP Scott}
%\affiliation{Chief Technology Officer, DuraChain, LLC}
% contact info here
%\date{\today}

%\begin{abstract}
%This paper outlines a practical application of an Electronic Healthcare Event environment developed for the providers of Durable Medical Equipment to be able to more efficiently provide care to their patients and get them the equipment that they need sooner. Additionally, we will be providing the details of how Electronic Healthcare Events can be developed to expand beyond the DME industry and make an impact throughout healthcare. It is the intent of this paper to showcase what EHEs can do for Healthcare if deployed properly and provided the right tools to implement them.
%\end{abstract}

%\maketitle

\section{Introduction}
We describe DuraChain, a decentralized Electronic Healthcare Event (EHE) environment. We will be utilizing the Matrix protocol to change the infrastructure of current Healthcare record keeping to provide a more secure, faster, and more accessible approach. We believe that, like us, our Healthcare data is a living thing that needs to be dynamically inclined to think patient forward. Furthermore, those who take care of us need a way to access this live data efficiently in an intuitive fashion. We will outline an all inclusive environment to solve this issue, propose an implementation plan for it, and provide details on the current market.
%
\subsection{Problem}
There are many problems being faced in healthcare today, of which we are prioritizing Durable Medical Equipment for this paper. Currently, it takes most DME providers approximately six months from starting to work with a patient until that patient receives their equipment. The reasons for this are due in large part to the availability of software that providers can access to help them streamline this process. Many providers are using upwards of five separate softwares that are poorly integrated and are spending upwards of \$15,000 per month for these services. Furthermore, research conducted has shown that patients and most smaller HCOs are unwilling to adapt to new technology that completely replaces what they are currently using, mainly due to not wishing to pay for such services. However, larger HCOs seem to be more open to new ideas and implementations. Therefore, we believe that the major issue with implementing patient-first technology is not going to help with the current needs of HCOs. There is a need for technology to be developed that rethinks how we can help fix the current issues with patient data as well as make it accessible to those who can impact the market directly the quickest.
%
\subsection{Solution}
We believe that the confounds lay in the ongoing trend toward patient-focused technology. (A prime example is the increasing frequency with which health systems are adopting "patient portals," which are almost universally slow and lacking elegant, intuitive design.) Far from believing that a patient should not have ready access and control over their health data, we argue that such an aim requires that technology for the stakeholders who routinely access, maintain, and transmit such information improves first.
%
Indeed, this technology is doomed to fail if conceived and developed in a vacuum; the needs of the patient must be front-of-mind; our solution stems from an approach that focuses on the provider and generates, updates, and transmits a patient's profile in a virtual representation of an exam room. Implementation of the Matrix protocol securely decentralizes patient data while offering an intuitive remedy to well-documented user interface issues in healthcare technology (HCT).
%
\section{Background}
DuraChain is a team with more than ten years of experience in the durable medical equipment and software engineering industries. In addition to being leaders in our industries, we have each been directly affected by the slow pace at which the healthcare industry moves. We have a particular sensitivity to patients who suffer from a lack of mobility and believe that our drive and subject matter expertise gives us a distinct edge.
%
\subsection{Durable Medical Equipment}
For most people, discussion of healthcare will probably trigger associations with hospitals and doctor's offices. Fewer people will draw connections to medical supplies and equipment that providers order for everyday or extended use. These products support the patient by providing mobility and independence.
%
Durable medical equipment is the set of products that are used on an ongoing basis to support a medical need and include hospital beds, oxygen concentrators and tanks, wheelchairs, crutches, commodes, ambulatory aids, blood glucose testing devices and supplies, and myriad other items.
%
  \subsubsection{Importance}
Durable medical equipment plays a crucial role in the modern healthcare system. Without the technology and supplies that make up DME, fatalities because of serious disease, complications, and even sleeping problems would surely rise. The National Association for Home Care reports that over 8 million people in the United States receive home care and values the industry at more than \$140 billion USD. With many populations around the world containing increasingly elderly cohorts, the trend indicates continued growth. From the emergency department to urgent care and assisted living facilities to inside the home, DME is present at every stage in the process of delivering healthcare.
%
\subsection{Distributed Ledger Technology}
Distributed ledger technology (\textbf{DLT}) replicates data across multiple devices based on consensus. These devices can be owned and operated by any person or institution so long as they have access to the network. Since there is no central locus of control over the data, both a peer-to-peer network and a robust consensus algorithm are necessary to ensure the accuracy and integrity of the data.
%
  \subsubsection{Misconceptions}
  Distributed ledger technology and blockchain technology are frequently confused as synonyms. Given the rapid onset of these technologies and how they took the attention the media and the public by storm, such confusion is understandable. Mere mention of a blockchain immediately points people's thoughts to financial transactions, speculative trading, imperfect markets defined by their volatility, and overnight millionaires.
%
  However, notions like these fail to capture the true scope and potential of DLT. To resolve any ambiguity, the only thing being recorded and transmitted across a distributed ledger is data. Any form of data storage can theoretically be retooled to use DLT. 
%
  The notion that DLT is an all-or-nothing model for software development is also a misconception. Numerous hypotheses circulate that claim DLT is the harbinger of unprecedented change and disruption to many industries. Over the long run, this may very well prove to be the case. There are, however, far more practical use cases ripe for implementation that will likely serve as stepping-stones since they can be realized quickly and with relative ease.
%
  In the healthcare domain, another persistent myth states that use a distributed ledger requires a radical redesign of how patient data is handled. While the current models are unequivocally ineffective, it is important to cultivate an awareness of what users are able to adapt to, use, and can ultimately result in a practical application.
%
  \subsubsection{Importance}
  With the level of security that is potentially available alone being a good enough reason to explore and test potential solutions, the added benefit of being able to reimagine how one might look at healthcare records to becoming a more patient-focused solution could put the original adopters of such technology years ahead of the field. The importance of DLT to the DME industry should actually be written as what the importance of the DME industry is to DLT. With the way that data is currently handled by the DME industry and with out in-depth understanding and experience in the DME industry, we see the potential for this industry to be the gateway to what the potential of DLT technology can be in Healthcare. What we are proposing is a gateway to a practical understanding and implementation of DLT in the Healthcare industry in a sector of the field that the technological implementation is possible, feasible, and will provide an easy enough transition to entice potential customers to utilize such a software in-addition to cutting their costs.
%
\section{Architecture}
The most important aspects of our technology are how data about patients will be stored and accessed by our users, who our users are, and how we plan to implement our software successfully. In-order to accomplish this, we must first outline what it is we are targeting to fix, explain how we’re doing so, and ensure those who implement our software that every aspect of it is fit for their patient base.
%
\subsection{Current Model}
The most commonly referred to methods of patient record keeping are the Electronic Healthcare Record (EHR) and Electronic Medical Record (EMR). These are centralized data collections on individual patients either centralized to a particular system or centralized to the single practice.
%
  \subsubsection{Electronic Healthcare Records}
  EHR or electronic health record are digital records of health information. They contain all the information you’d find in a paper chart — and a lot more. EHRs include past medical history, vital signs, progress notes, diagnoses, medications, immunization dates, allergies, lab data and imaging reports. They can also contain other relevant information, such as insurance information, demographic data, and even data imported from personal wellness devices.
%
  \subsubsection{Electronic Medical Records}
  The EMR or electronic medical record refers to everything you’d find in a paper chart, such as medical history, diagnoses, medications, immunization dates, allergies. While EMRs work well within a practice, they’re limited because they don’t easily travel outside the practice. In fact, the patient’s medical record might even have to be printed out and mailed for another provider to see it.
%
\subsection{DuraChain Model}
The EHR is an evolution of the EMR, however, despite it’s positive direction, it is still limited in the way they are being handled and implemented. A less centralized collection of this data and a ledger of activity can be achieved by decentralizing the data amongst those who access the data about the patient. The DuraChain model will be approaching healthcare as if each patient is in their own room, just like a hospital visit in real life.


%
The EHR in nature is not what we intend to change. We do believe that an EHR for a patient is a critical shift to better patient care. However, what we insist is that the data be stored in a more suitable fashion that is less about the data being worth monetary value. To do this, we must consider the patient the most important, invaluable piece of data, and as such, no one entity can “own” the data about them. However, through permission based inviting, multiple entities from separate organizations may update patient records and be aware of updates made by others. Being aware of what is being updated about a patient will be known as Electronic Healthcare Events.
%
  \subsubsection{Electronic Healthcare Events}
  An Electronic Healthcare Event (EHE) is an update to a patient’s EHR ledger. An EHE is sent to a Patient Room and once confirmed, the EHE is made visible on the patient’s ledger and an update is made to their EHR reflecting the event.
%
  Utilizing the technology advancements that Matrix has made with Event based data storage, we will decentralize not only the records of the patient but as well as the conversations about the patient. This will allow for a never before look and understanding of patient forward delivery of DME that will save time and provide a more streamlined approach to putting the equipment in the patient's home. Additionally, it allows for scalability of the software beyond the DME industry in further implementations.
%
    \paragraph{Benefits of Event-Based Distribution}
    In an event based distribution model we can interact in real time without a central authority controlling the data being posted, but a collective understanding of the functionality of the space we locate ourselves in. In our case, we are focused on providing the data of a user. Currently, many in the industry are worried only about dynamic content posting, live interaction so to speak. However, Healthcare is a great example of where static content posting is still alive and well. In this sense, what our system allows for is a dynamic understanding of the change happening to static content. While certain variables about a patient may technically be static, such as their name, by making these things events we can track who updates what details about patients and thusly use this data about who is active inside of the room as a method of understanding the patient care process better.
%
    \paragraph{Syanpse Firing Analogy}
    Each server is setup in such a way that the collective works much like the synapse of our brains. The different servers fire events off that ripple throughout the synapse and post to collective spaces to let the other servers know it’s done so. This is the best way to illustrate how events are taking place. A server will take an event posted by a user fire it off to the rest of the servers and the room will confirm the fire by posting the event.
%
  \subsubsection{Patient Rooms}
  Each patient will be added to the software by the creation of a “room” for that patient. Here, the EHE will be posted by the various users and the most up-to-date information about the patient will be the consensus shown. Inside of these rooms will be a running ledger of EHE posting joined with discussion about the patient that may be necessary between the users inside of the patient’s room.
%
  The details available on a Patient will be events posted to the room. The events that will be posted by and visible to the users of DuraChain about the patient will be:
%
  \texttt{
  {

  firstName,

  middleName,

  lastName,
  primaryInsProvider,
  primaryInsIdentifier,

  secondaryInsProvider,

  secondaryInsIdentifier,

  height,

  weight,

  gender,

  homePhone,

  mobilePhone,

  homeAddress
  }
  }
%
  \subsubsection{Users}
  User permissions within a Patient’s Room is one of the most critical aspects involved in ensuring the security of patient data. That is why we have developed a system where Users are permitted to a Patient’s Room only by invitation of a qualified member of that Patient’s Room already. In the event the Patient does not have a Room already built, a qualified permission user associated with the Provider working with that Patient will be able to initiate the creation of a Patient Room.
%
  \paragraph{Administration}
  Administrative users are the engineers and developers who work on the DuraChain project. While they will not have access to any Patient Room directly, they are responsible for the development and maintenance of the source code associated with the project. An administrator has no access to nor knowledge of any room they are not invited to, further doubling down on security, however, they may be invited to any Patient Room in-order to conduct any updates/maintenance or debug issues that arise for our clients.
%
  %\paragraph{Sales Representative}
  % Sales Representative is a certified Assistive Technology Professional (ATP) through Resna which allows them to complete the necessary forms for patients. These users will be in-charge of higher tier facilities in their area and be able to create a Patient Room when appropriate. These users will be able to post EHEs to a patient’s EHR ledger & perform many of the tasks available within the Patient Room.
%
  %\paragraph{Overlay Representative}
  %An Overlay Representative is a less experienced version of a Sales Representative. They will often take lower tier facilities in the area and works with less complicated DME. They will have a Sales Representative who will be working with them that will need to confirm any EHEs they may post to a a patient’s EHR ledger. As they progress and become ATP certified, these users end up becoming Sales Representatives.
%
\subsubsection{Facilities}
To better navigate the vast amount of patients rooms that will be created on the network, we are utilizing the Group function built into the Matrix protocol to be able to organize Patients by facility. A user who is responsible for managing facilities and patients within said facility can add Patient Rooms to a facility for better organization of the patients and a more streamline approach to understanding our UXUI. It is our goal with our software to not only reimagine how EHRs can be handled and stored, but as well as providing a balance with the most approachable UXUI in the industry. By providing the facility function, patients are organized in a way so that the people who need to quickly sort through the patients they’re working with can do so.
%
\subsection{implementation}
Implementation of the DuraChain environment will require participation by providers of DME, whom will be our main client base. In-order to properly bring a client into our environment, they must have a copy of our server installed and properly attached to their domain. From there, they can access our environment through our custom-built UI and begin creating Patient Rooms. In-order to join an already created patient room, they must be invited to that room by someone with credentials inside of that room to do so.
%
\subsubsection{Protected Health Information}
There is absolutely a need to discuss the topic of Protected Health Information (PHI) when dealing with sensitive data such as Healthcare. When discussing this for the implementation of the DuraChain environment, what we established was that the servers with the data about the patients and the users who access this must all be HIPPA compliant in-order for our software to be implementable. So, by placing the servers in the domain of the DME provider and the user creation in their hands as well, we are only responsible for ensuring that the clients we bring on to access the DuraChain insure us of their HIPPA compliance, which means, in short, we have put this responsibility in the hands those who implement our software. Not to say that we wish to push compliance issues onto others, but rather, those who are the user base of DuraChain were already needing to be HIPPA compliant in their business already. Therefore, so long as our customer is compliant, so is our software.
%
\section{UXUI}
We have outlined throughout this paper how a radical redesign of the EHR is not the first or the most practical implementation of decentralized software into the Healthcare industry. However, we do believe that in-order to combat the massive issues faced by the industry in-terms of software design a radical change must happen. The most radical change should take place in the User Experience & User Interface (UXUI) of the software that is accessible the healthcare records to be more clear on the data being changed, easier to implement in HCO’s, easier to train new users to use, and to be Accessibility friendly.
%
\subsection{UXUI Overhaul}
The best way to approach the UXUI is for the people interacting with the patients' data on a daily basis. So, what we’ve done is we’ve sat down with some of the leading providers of DME in the industry and figured out what would be the best way to implement a UXUI that gives their technicians a shorter learning curve and provides a substantial enough price incentive to switch from their current EHR software.
%
\subsection{Accessibility}
One of the growing issues in software development, particularly the UXUI field is accessibility. Software Accessibility has two priorities as outlined by the Centre for Excellence in Universal Design. Priority 1 being to ensure “that the application can be used by most people with impaired mobility, vision, hearing, cognition and language understanding, using their assistive technologies,” and Priority 2 being to “make it easier to use and will include more people with cognitive impairments or multiple disabilities.” In the scope of Healthcare UXUI there is not many applications to-date that can tout Software Accessibility and it is becoming more and more of a rising concern. From our research, if one looks at the higher earning organization you’ll begin to see them paying more and more for additional software to integrate into their already clunky conglomerate of slow, under performing, and overpriced software. Additionally, you see the lesser earning organizations opting for whatever is cheapest and within reach, which are still often pricy. In all of these cases we see a real lack of Accessibility within an industry that we believe this to be an absolute necessity. Which is why the development of our UXUI will be fully compliant with all the necessary requirements for Accessible Software.
%
\subsection{Implementation}
We plan to implement such a massive overhaul to the UXUI issue surrounding Healthcare by providing a lightweight javascript application to access our environment that can be accessed from any web browser. We plan to implement an application for the iPhone and Android operating systems as well. Additionally, we will be making our UXUI independent of the data that is being accessed. That is, simply by signing into the platform isn’t enough. A user must be invited to a particular Patient’s Room. This is important for not only security, but speed of the application as well. An application independent of the data is not typical in today’s Healthcare market and is why the DuraChain environment will be able to maintain it’s speed while being able to bring on large quantities of users that DME providers need.
%
\section{Market}
Another big aspect in the success of the DuraChain environment is ensuring that we target the proper market to implement our strategy and provide them a solution that is innovative, but also practical to use. Additionally, we must know what has come before us and what is currently available on the market to be able to best assess the demands of the target market to deploy our environment in the maximally efficient manner. We are not the first to envision a world where patient data is transformed into something more useful, practical, and accessible. However, we are the first to approach it in a manner that targets DME providers, a very data-forward sector with a need for such a software solution.
%
\subsection{Target Market}
Our target market are the providers of DME equipment. Studies show that What we aim to show them is that we can reduce their costs, addresses the needs to unify software, and provide an easy to implement solution. As previously mentioned, our target market is currently paying upwards of \$5,000-15,000 per month for access to three or more pieces of software.
%
  \subsubsection{Needs and Wants}
  Our target market is looking for a product that puts the power of the current software out there into an all inclusive environment and not have to pay such extreme prices to achieve this. Many of the DME equipment providers out there are using software that takes noticeable time to load and don’t often integrate the way they planned.
%
  One potential customer we sat down with expressed with great dissatisfaction how he was paying for a particular platform’s metric system but also needing to maintain a spreadsheet of these particular metrics that they update manually. When questioned about this, it appeared the issue fell on the need for particular fields to be filled out in one program in-order for the metric system to do any of it’s tasks. Addressing issues like this is among one of the many ways our system will be able to be the solution providers need while advancing the technology behind healthcare data.
%
  \subsubsection{DME provider Groups}
  There are a few different types of DME providers, \$0-5m , \$10-20m , and the whales of the industry \$25-60m. Each has a group different way of providing, amount of time they take for delivery, likelihood of joining our platform and what they might use it for. Outlined below are the definitions of each of these groups.
%
  \subsubsection{\$0-5m}
  A \$0-5m is a provider that: 1) utilizes manual documentation and billing instead of paying \$6,000 a month for outdated systems; 2) Has longer documentation time and time of supply the patient with their DME equipment; and 3) has a minimal barrier to entry. On average a provider in this group will pay \$2,000-4,000 for their software needs.
%
  \subsubsection{\$10-20m}
  A \$10-20m is provider who licenses at least 3 different programs for everyday business and will rarely use manual documentation reporting. \$10-20m providers spend over \$10,000 per month on these applications to keep track of orders, documentation, and to bill insurances.
%
  \subsubsection{\$25-60m}
  Being a \$25-60m providers is all about the revenue per month to license anywhere from 5 to 8 programs for daily use. Being a provider in this category you are spending over \$20,000 per month just to keep your production where it is. The providers in this category don’t suffer from human error the way the others do but their biggest problem is data loss going from program to program just for one order.
%
\subsection{Competition}
The definition of who our competitor is may be up to interpretation at this point, but we believe that our main competitors are within the DME software industry and beyond that, Healthcare software. We do not aim to compete with other decentralized technologies inherently, we aim to work with these types of people to better our EHE system and focus our monetization via our UXUI.
%
  \subsubsection{Brightree}
  Brightree is a DME billing system that sold for \$800M in 2016 to ResMed, a respiratory company. What is beneficial about their software is that they already have mobile application functionality built in, a Medicare and Medicaid pricing table that updates, and a system in place that connect with referral sources, a Physical Therapist referring a patient, for example. Additionally, they work with you step-by-step to provide a comprehensive custom build of their environment.
%
  However, on the negative we notice that Brightree falls short in documentation, UXUI, and implementation time. The User Interface was created in 2008 and has yet to have an update. It lacks ease-of-use and has many features but is only really used for billing insurances. This parleys into the documentation issue, causing it to be slow which leads to often outdated data. They haven’t seemed to need to update their software because they have spent \$140,000,000 over the past seven years acquiring competitors and their 2,300 users, choking the market.
%
  \subsubsection{Lazarus}
  Lazarus is a patient documentation system created by Orbit Medical. It has a very usable task flow system and provides a very workable note-taking system to it’s users. Their system allows for patient attachments to be uploaded to their file, track sales representatives, equipment deliveries, and any repairs made to the equipment.
%
  Despite allowing for uploading of patient documentation, there is often duplicates of these files in their system. Additionally, the search criteria is limited only to \texttt{orderNumber firstName, lastName} of patient. They struggle from outdated UI as well, being created before 2010 and when accessing a patient sales order through this UI only one user can be inside of their file at a time - limiting the support and efficiency of the process. Their integration is limited, most notably that you must add a new product to Lazarus although the product may be in added to another software such as Iboss and implementation of their software takes 6 to 12 months per provider.
%
  \subsubsection{Iboss}
  Iboss is an inventory system for DME providers. They store products via serial number and track each item by manufacturer item number. They don’t, however, pair well with other software solutions necessary for the target market, such as not inputing inventory into Lazarus. Like most of the software currently available there is an issue with duplication and time for implementation. Additionally, this system only allows search by serial number. Since the system is mostly removed from other software, this makes tracking the inventory associated with a patient more difficult.
%
  \subsubsection{Domo}
  Domo is a data reporting company that has found great success, raising nearly \$700M in seven years. They have internal messaging, app stores to purchase cards, and a nicely designed front-end client for browsers. They suffer from less that 100\% reporting and a slow system (graphs don’t populate properly when launch). Additionally, they have a steep fee for usage of \$83 per user per month, which adds up quickly in a market where one-hundred field representatives isn’t an unrealistic number of users.
%
  \subsubsection{Health Splash}
  Health Splash is a patient documentation and physician system that states it will be using blockchain technology by the third quarter of 2018. They will be launching a mobile application that connects patients and doctors to check insurance, set appointments, check ER wait times, and chat directly with doctors. A patient will be able access their own medical records, view them, and send them wherever they so choose.
%
  The potential upside of Health Splash is huge, but there is little known about how they will be implementing blockchain technology. What they have provided online is questionable in terms of credibility in the blockchain space, as much of it is evidently a reiteration of easily definable terms found from a Google search. Health Splash plans to do an Initial Coin Offering, which will offer us some insight into their future goals.
%
\section{Conclusion}
DuraChain will help DME providers be able to provide more efficient patient care with an innovated, all-in-one solution that decreases their costs and the time it takes for a patient to go from needing DME to receiving it. We’ve shown how DuraChain can be scaled beyond the DME industry and used by everyone involved in the patient care process. If we focus on the patient and provide those who take care of the patient the fastest, most accurate, and most secure way of interacting with the patients healthcare data we can fix the longstanding and still growing issues surrounding it.
%
%————————————————————————
% In this section, "Add a paragraph about DuraChain like the competition layout on previous page. Show our pros to their cons."
% In this section, add table comparing DC to competition
%————————————————————————
%
\subsection{Next Steps}
In-order for DuraChain to work properly, the system needs DME providers to use the environment instead of our competitors. To best position ourselves against our competition, we will be implementing a test environment that a select chosen providers will use in a demo mode to gain feedback before deploying on a larger scale. Once we’ve gone through sufficient testing, roll out of the system on a larger scale will be handled by our internal Sales Department.
%
\subsection{Future}
DuraChain is here first to fix the issues that are facing the DME industry today but our plan is to take on the larger issue of healthcare data and become an integral part of the patient care process. We believe that doing so involves approaching healthcare data as putting the most power in the hands of the people taking care of us. Through the details proposed within this paper we believe that DuraChain can be the next step in patient healthcare data.
%
\begin{acknowledgments}
%
\end{acknowledgments}


\appendix*
\section{More Information}
%
\appendix*
\section{Key Terms}
%
\begin{thebibliography}{99}
%
\bibitem{OED} Oxford English Dictionary
%
\bibitem{source2} Second Source.
%
\bibitem{Comment} This is an example of a comment that may be necessary in this
document. These are usually about subtle, finer points
that may be necessary to declare or explain but that may distract from
the body text.
%
\end{thebibliography}
\end{document}
