\section{Market}
A thorough assessment of the market is another key component of the success of the DuraChain environment. We are focused on targeting the proper market segment and providing it with an innovative, practical solution.%

Software development and deployment does not exist in a vacuum, and it is therefore important to accurately assess the demands of the target segment to develop an understanding of what products they are currently using and have used previously. A fine-grained understanding of these details will allow us to deploy our own environment in the maximally-efficient manner.%

We are not the first to envision a world where patient data is transformed into something more useful, practical, and accessible, we are the first to take an approach that targets DME providers with a data-forward solution---something the segment desperately needs.%

\subsection{Target Market}
We are presently targeting durable medical equipment providers. We aim to reduce their costs, unify their software environment, and do so with an easy-to-implement solution.%

As mentioned above, \textsc{dme} providers are currently paying \$5,000-15,000 per month for access to three or more pieces of software.%

  \subsubsection{Needs and Wants}
  Many \textsc{dme} providers are using software that suffers from significant load times and fails to neatly integrate with the rest of their environment. They are hungry for a product that puts the power of existing software into a homogenous environment at a reasonable cost.%

  In an interview, a potential customer expressed his dissatisfaction that he is paying for access to the metric-tracking system of a particular platform but also had to manually maintain a separate spreadsheet of this data. This unfortunate circumstance stems from the fact that program requires data to be clumsily entered into fields in order for it to actually execute any metrics-tracking tasks.%

  Gaps like these are where DuraChain will be able to fill the needs and desires of \textsc{dme} providers while simultaneously advancing healthcare data management by leaps and bounds.%

  \subsubsection{DME Provider Groups}
  Durable medical equipment companies can be sorted into brackets according to their gross revenue. The brackets differ in their approach to providing equipment, typical delivery times, likelihood of adopting the DuraChain platform, and how they might leverage its capabilities.%

  \subsubsection{\$0-5,000,000 Gross Revenue}
  This class of \textsc{dme} provider is characterized by:%
  \begin{itemize}
    \item manual documentation and billing procedures;
    \item prolonged documentation times;
    \item an extended time frame between first contact with the patient and final delivery of their \textsc{dme}; and
    \item a minimal barrier to entry.
  \end{itemize}
  A provider in this revenue bracket will typically spend \$2-4,000 each month for their software.%

  \subsubsection{\$10-20,000,000 Gross Revenue}
  A \textsc{dme} provider in this bracket will:%
  \begin{itemize}
    \item license at least 3 different softwares for business operations; and
    \item rarely generate documentation manually.
  \end{itemize}
  This class of provider spends over \$10,000 per month on software to track sales orders, generate and transmit documentation, and bill to insurance carriers.%

  \subsubsection{\$25-60,00,000 Gross Revenue}
  \textsc{dme} providers in this revenue bracket are more or less at the top of the food chain. They tend to:%
  \begin{itemize}
    \item license 5 to 8 programs for daily use;
    \item
    \item
  \end{itemize}
  \textsc{dme} providers in this top bracket can expect to pay \$20,000 or more per month for software that barely allows them to maintain stable sales and production levels.%

  Because of the extensive use of software in this bracket, human error is greatly reduced, but these providers also routinely suffer from data loss as workflow moves between softwares.%

\subsection{Competition}
Given the unique nature of DuraChain's solution, it is hard to identify any single entity that directly intersects with the full range of what we offer. Generally speaking, we view our closest competitors as software companies broadly specializing in healthcare data or with a particular emphasis on DME.%

Notably, we do not perceive vendors of other forms of decentralized technology as direct competitors. In fact, we aim to foster collaboration with these organizations to improve our \textsc{ehe} system while we focus our monetization efforts on our \textsc{uxui}.%

A complete comparison between DuraChain and the competitors listed below can be found in table \ref{tab:competition}.

  \subsubsection{Brightree}
  Brightree is a \textsc{dme} billing system that sold to respiratory company ResMed in 2016 for \$800,000,000. Notable aspects of the Brightree platform include a mobile app, MediCare and MedicAid pricing tables, and the ability to connect with referring medical providers. They also work directly with clients to implement custom builds of the Brightree environment.%

  Brightree falls short in a number of ways, though. The \textsc{ui} is 10 years old and has never been updated. Technically packed with features, the difficulty of using the software means that most \textsc{dme} companies only use Brightree for billing. The software is also very slow and frequently contains outdated data.%

  Instead of proactively addressing these deficiencies, Brightree has instead spent \$140,000,000 in the last 7 years to acquire competitors and corner the market with their 2,300 users.%

  \subsubsection{Lazarus}
  Lazarus is a patient documentation system created by Orbit Medical. Workflow is usable and it offers a good note-taking system. The system allows patient documentation to be attached to a patient file. It can also track sales representatives, deliveries, and repairs made to \textsc{dme}.%

  The patient documentation system often contains duplicates, however. Search criteria are limited to \texttt{orderNumber}, \texttt{firstName}, and \texttt{lastName} of the patient. The Lazarus \textsc{ui} is also outdated, having been created sometime before 2010. The \textsc{ui} only allows a single user access a patient file at a time, imposing severe constraints on efficiency. Integration is limited and implementation can take from 6 to 12 months for each \textsc{dme} provider.%

  \subsubsection{Iboss}
  Iboss is an inventory system for \textsc{dme} providers. It stores products via serial number and tracks each item by manufacturer SKU, but search criteria are limited to serial numbers. Integration with other solutions required by \textsc{dme} providers is limited. For example, Iboss cannot export inventory data to Lazarus. Since Iboss exists in relative isolation to other \textsc{dme} software products, tracking inventory associated with a patient is difficult.%

  \subsubsection{Domo}
  Domo is a data reporting company that has found great success, raising nearly \$700,000,000 in seven years. They have internal messaging, app stores to purchase cards, and a well-designed front-end client for browsers. They suffer from incomplete reporting and a slow system. Additionally, they have a steep fee of \$83 per user per month, which adds up quickly in a market where 100 field representatives is not unheard of.%

  \subsubsection{Health Splash}
  Health Splash is a patient documentation and physician system that states it will be using blockchain technology in 3Q18. They intend to launch a mobile application that connects patients and doctors to check insurance, set appointments, check ER wait times, and chat directly with doctors. A patient will be able view their medical records and send them wherever they choose.%

  The potential upside of Health Splash is huge, but little is known about how they will be implementing blockchain technology. The information they provide online is questionable---much of it clearly a restatement of terms and buzzwords easily found with a Google search. Health Splash plans to conduct an Initial Coin Offering (\textsc{ico}), which may offer more insight into their future goals and ability to execute.%

\subfile{./subfiles/tables/competition}
