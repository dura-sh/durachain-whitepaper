
\section{Introduction}
We describe DuraChain, a decentralized electronic healthcare event (\textbf{EHE}) environment. Using the Matrix protocol,
\footnote{\textit{cf.} \url{https://matrix.org/blog/home/}}
we will change the infrastructure of healthcare record-keeping by providing an approach that is faster, more secure, and more accessible. We adopt the position that healthcare data is a living entity that mandates a dynamic, sustainable environment in order to thrive and best serve the needs of patients. Following on from that, providers and other stakeholders in the healthcare industry need an efficient and intuitive way to access this data in real time. To address this issue, we outline a comprehensive solution, propose our implementation, and provide a brief analysis of the current state of the market.%

\subsection{Problem}
While many problems exist in the healthcare industry, we prioritize durable medical equipment (\textbf{DME}) at our current stage of development. At present, it takes most DME providers approximately six months from initial contact with a patient to deliver necessary equipment that improves their quality of life. In large part, this lag is due to the poor availability and/or quality of software that DME providers use to facilitate the flow of information throughout the DME sales order life cycle. In our view, six months is entirely too long.%

Many DME providers use five or more separate softwares that suffer from poor integration and cost in excess \$15,000 per month (\$180,000 per annum), not including labor costs. Many patients and most small-scale healthcare organizations (\textbf{HCO}s) are unwilling to adopt new technology that has the potential to replace their current tools---mainly because of an unwillingness to carry the cost of such services. Larger HCOs, however, seem to be more receptive to new ideas and implementations.%

Thus, we believe that an adequate implementation of patient-first technology that simultaneously addresses the needs of HCOs and patient requires a wholesale rethinking of patient data management so as to make it accessible to those parties who can rapidly impact the market.%

\subsection{Solution}
We believe that the confounds lay in the ongoing trend toward patient-focused technology. (A prime example is the increasing frequency with which health systems are adopting ``patient portals,'' which are almost universally slow and lacking elegant, intuitive design.) Far from believing that a patient should not have ready access and control over their health data, we argue that such an aim requires that technology for the stakeholders who routinely access, maintain, and transmit such information improves first.%

Indeed, this technology is doomed to fail if conceived and developed in a vacuum; the needs of the patient must be front-of-mind; our solution stems from an approach that focuses on the provider and generates, updates, and transmits a patient's profile in a virtual representation of an exam room. Implementation of the Matrix protocol securely decentralizes patient data while offering an intuitive remedy to well-documented user interface issues in healthcare technology (HCT).%
