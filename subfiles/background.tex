
\section{Background}
DuraChain was conceptualized by a team who have a combined 30+ years of experience in durable medical equipment and software engineering industries. In addition to being leaders in our industries, we have each been directly affected by the slow pace at which the healthcare industry moves. We have a particular sensitivity to patients who suffer from a lack of mobility and believe that our drive and subject matter expertise gives us a distinct edge.%

\subsection{Durable Medical Equipment}
For most people, discussion of healthcare will probably trigger associations with hospitals and doctor's offices. Fewer people will draw connections to medical supplies and equipment that providers order for everyday or extended use. These products support the patient by providing mobility and independence.%

Durable medical equipment (\textsc{dme} is the set of products that are used on an ongoing basis to support a medical need and include hospital beds, oxygen concentrators and tanks, wheelchairs, crutches, commodes, ambulatory aids, blood glucose testing devices and supplies, and myriad other items.\cite{dmeDefinition}%

\subsubsection{Importance}
Durable medical equipment plays a crucial role in the modern healthcare system. Without the technology and supplies that make up \textsc{dme}, fatalities because of serious disease, complications, and even sleeping problems would surely rise. The National Association for Home Care reports that over 12 million people in the United States receive home care and values the industry at more than \$40 billion USD.\cite{homecarestats}\cite{dmemarket} With many populations around the world containing increasingly elderly cohorts, the trend indicates continued growth. From the emergency department and urgent care to assisted living facilities and even inside the home, \textsc{dme} is present at every stage in the process of delivering healthcare.%

\subsection{Distributed Ledger Technology}
Distributed ledger technology (\textsc{dlt}) replicates data across multiple devices based on consensus. These devices can be owned and operated by any person or institution so long as they have access to the network. Since there is no central locus of control over the data, both a peer-to-peer network and a robust consensus algorithm are necessary to ensure the accuracy and integrity of the data.\cite{DLT}%

  \subsubsection{Misconceptions}
  Distributed ledger technology and blockchain technology are frequently confused as synonyms. Given the rapid onset of these technologies and how they took the attention of the media and the public by storm, such confusion is understandable. Mere mention of a blockchain immediately points people's thoughts to financial transactions, speculative trading, imperfect markets defined by their volatility, and overnight millionaires.%

  However, notions like these fail to capture the true scope and potential of \textsc{dlt}. To resolve any ambiguity, the only thing being recorded and transmitted across a distributed ledger is data. Any form of data storage can theoretically be retooled to use \textsc{dlt}.%

  The notion that \textsc{dlt} is an all-or-nothing model for software development is also a misconception. Numerous hypotheses circulate that claim \textsc{dlt} is the harbinger of unprecedented change and disruption to many industries. Over the long run, this may very well prove to be the case. There are, however, far more practical use cases ripe for implementation that will likely serve as stepping-stones since they can be realized quickly and with relative ease.%

  In the healthcare domain, another persistent myth states that use of a distributed ledger requires a radical redesign of how patient data is handled. While the current models are unequivocally ineffective, it is important to cultivate an awareness of what users are able to adapt and use, which can ultimately result in a practical application.%

  \subsubsection{Importance}
  Forecasting the impact of a new technology on any particular industry is a challenging task. That being said, it may be prudent to frame the \textsc{dme}-\textsc{dlt} interface as membrane: While the positive effects of \textsc{dlt} on the \textsc{dme} space are easy to see, implementation of distributed ledger technology in a flagship industry like \textsc{dme} and healthcare data may provoke a sea change in a number areas.%

  Potential improvements in security alone provide a compelling reason to test solutions. What's more, the ability to generate, manage, and transmit records (healthcare or otherwise) with a greater focus on the customer is a non-trivial benefit and could put early adopters of such technology years ahead of the rest of their field.%

\subsection{Combining DME and DLT}
  At the moment, \textsc{dlt} and \textsc{dme} exist in relative isolation. Data handling in \textsc{dme} is demonstrably weak and healthcare is opaque to leaders in \textsc{dlt}. A marriage between the two fields is sure to be profitable for the technology, healthcare stakeholders of all types, and a potentially limitless number of other domains.%

  What we propose here is meant to serve as the gateway to a practical understanding and implementation of \textsc{dlt} in healthcare. Our choice to focus narrowly on \textsc{dme} does not reflect a limited scope of thought but rather a calculated targeting of an implementation that is possible, feasible, and promotes an easy transition for customers while cutting their costs.%
