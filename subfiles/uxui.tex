\section{UXUI}
Throughout this paper, we have argued that a wholesale redesign of the \textsc{ehr} is not the optimal strategy for implementing decentralized software in the healthcare industry. Instead, we believe that the element of healthcare data management that cries out most for a radical rethinking is software design. Success in this domain will make great strides in the effort to combat the massive issues facing healthcare stakeholders.%

The most radical changes should occur with respect to the user experience and the user interface. Users will benefit greatly from being able to clearly understand what data in an \textsc{ehr} has been changed, the straightforward implementation of our system, the ease with which new users can be trained on the software, and the improvements in accessibility.%

\subsection{UXUI Overhaul}
Rethinking the \textsc{uxui} of \textsc{ehr} software requires an approach that prioritizes the needs of the people who interact with patient data on a daily basis. We have conducted interviews with several of the leading providers of DME in an effort to determine the best to implement a \textsc{uxui} that provides a shorter learning curve than existing software while incentivizing \textsc{dme} companies to make the switch with competitive pricing.%

\subsection{Accessibility}
Software developers are becoming increasingly concerned with ensuring that their products adhere to accessibility best practices.\cite{w3Accessibility} This is particularly important in the areas of user experience and user interface.%

The Centre for Excellence in Universal Design\cite{universaldesign} promotes two priorities for accessible software designs:%
  \begin{itemize}
    \item \textbf{Priority 1} requires that ``that the application can be used by most people with impaired mobility, vision, hearing, cognition and language understanding, using their assistive technologies;'' and
    \item \textbf{Priority 2} is to ``make [software] easier to use and will include more people with cognitive impairments or multiple disabilities.''
  \end{itemize}
To date, there are very few, if any, healthcare applications that can claim to closely adhere to these principles.%

In our research, we found that top-grossing \textsc{hco}s are laying out increasing amounts of capital to integrate a growing number of inadequate programs into their already-clunky amalgam of slow and under-performing software. By contrast, lower-revenue \textsc{hco}s are inclined to choose the cheapest products within reach---and even these ``standard'' options contribute significantly to the SG\&A line item.%

In either case, we see a distinct lack of attention paid to accessibility. For this to be the case in a industry where such design concerns are likely more critical than any other, we believe a renewed emphasis on accessibility is an absolute necessity.%

Bearing all of this in mind, we are fully-committed to adhering to best practices for software accessibility.%

\subsection{Implementation}
A lightweight Javascript application is our vehicle of choice to undertake such a massive overhaul of the prevailing \textsc{uxui} standards in healthcare software. The application can be accessed from any modern web browser. Mobile apps for iPhone and Android operating systems will complete our software lineup.%

Additionally, we will also implement our \textsc{uxui} independently of the data that is being accessed. That is, just singing in to the platform enough to begin reading or writing data---a user must also be granted permission to access a Patient Room. This is important for security, of course, but also helps us optimize the application's speed.%

This Chinese wall is atypical in the software marketed to \textsc{hco}s today. Indeed, a significant portion of DuraChain's edge is derived from the fact that our implementation will ensure a fast, lightweight application that is also capable of handling the large quantities of data that \textsc{dme} providers need.%
