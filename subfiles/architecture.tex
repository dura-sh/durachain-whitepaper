
\section{Architecture}
The foundation of our technology is data generation, management, and transmission. These cornerstones include novel ways of storing patient data, identifying and providing access to various types of users, and our overall software implementation.%

To accomplish these objectives, we have carefully considered the precise target of our solution, our plan for implementation and deployment, and a process to ensure that our software adequately addresses the needs of providers and their patient base.%

\subsection{Current Model}
The most common methods of maintaining a database of patient records are the electronic healthcare record (\textsc{ehr}) and the electronic medical record (\textsc{emr}). Both \textsc{ehr} and \textsc{emr} are centralized collections of data pertaining to an individual patient. They may be centralized to a particular system or a single healthcare practice.%

  \subsubsection{Electronic Healthcare Records}
  An electronic health record (\textsc{ehr3}) is a digital record of health information. It contains all of the information found in a typical paper chart---plus a lot more. \textsc{ehr}s include medical history, vital signs, progress notes, diagnoses, prescribed medications, immunization dates, allergies, lab orders and results, and medical imaging data. An \textsc{ehr} may also contain other relevant information pertaining to health insurance, demographics, and even data imported from personal devices.\cite{EMRvsEHR}%

  \subsubsection{Electronic Medical Records}
  An electronic medical record (\textsc{emr}) is a more limited form of health information. It contains all of the information found in a typical paper chart. \textsc{emr}s include medical history, vital signs, progress notes, diagnoses, prescribed medications, immunization dates, allergies, lab orders and results, and medical imaging data. While \textsc{emr}s work well within the confines of a single practice, their utility is limited by the fact that they cannot readily travel outside that practice. In fact, a patient's \textsc{emr} must often be printed and mailed or faxed in order for an outside provider to access it.\cite{EMRvsEHR}%

\subsection{DuraChain Model}
The \textsc{ehr} is an enhanced version of the \textsc{emr}. Despite the apparent movement in a positive direction, implementation and handling of \textsc{ehr}s still impose severe limitations.\cite{palmaEHR} By standardizing and decentralizing this data and storing it in a ledger, a greater number of stakeholders can securely access patient data. DuraChain's approach to healthcare data places each patient and their data in their data into a room analogous to a real-life hospital visit.%

The \textsc{ehr} represents a critical and positive shift toward better patient care and we do not intend to alter it. Instead, we insist that the data be stored in a manner that places less emphasis on the monetary value of the data.%

We achieve this by regarding the patient as the single most important, invaluable piece of data. As such, no single entity ought to be allowed to ``own'' the data about a patient. Through invitations and permissions-based access, multiple stakeholders in a patient's care may view and update patient records while being made aware of updates made by other parties. We term the process of receiving updates about a patient an Electronic Healthcare Event (\textsc{ehe}).%

  \subsubsection{Electronic Healthcare Events}
  An Electronic Healthcare Event (\textsc{EHE}) is an update to a patient’s \textsc{ehr} ledger. An \textsc{ehe} is sent to a Patient Room and once confirmed, the \textsc{ehe} is made visible on the patient’s ledger and an update is made to their \textsc{ehe} reflecting the event.%

  Advances in event-based storage made by Matrix allow us to decentralize both records an conversations about a patient. This permits an unprecedented and patient-forward \textsc{dme} sales order life cycle process that will save time and streamline the approach for getting equipment into the patient's home in a timely manner.%

  Event-based storage also scales readily, allowing for rapid, modular development and implementation of softwares beyond the DME application discussed here.%

    \paragraph{Benefits of Event-Based Distribution}
    Using an event-based distribution model, stakeholders can interact in real time without the need to communicate with any of the centralized authorities that presently act as gatekeepers and bottlenecks. The real-time functional and spatial awareness enjoyed by all parties will lead to a greater understanding of where, when, and in what state any given sales order exists in its life cycle.%

    For our present \textsc{dme} implementation of \textsc{dlt}, our chief concern is providing data about the user(s). Currently, many actors in the \textsc{dlt} space are only concerned with dynamic content posting---live interaction, so to speak. Healthcare, however, is an excellent example of where static content posting is still alive and well.%

    In this regard, our system allows for a dynamic understanding of change as it acts on ``static'' content. Certain variables about a patient, like their name, can certainly be viewed as static. By making these variables event, though, we can track who updates what details about a patient and thus use these data to track who is active inside a patient's room at any given time. The fine granularity offered by event-based information distribution will foster a deep and improved understanding of the patient care process.%

    \paragraph{Servers as Synapses}
    A useful analogy to conceptualize how the network of servers pass information back and forth is to imagine each connection as a synapse (the interface between neurons) in the brain. Neurons receive a stimulus and respond to that stimulus by taking an appropriate action or simply passing the input ``down the line.''%

    In our network, a server will take an event posted by a user and fire it off to the rest of the servers it shares a connection with. The collective network then provides feedback about the broadcast by confirming and posting the event. Different servers fire off events that ripple throughout network of synapses and post to common spaces to inform the other servers that they have done so.%

  \subsubsection{Patient Rooms}
  Patients are added to the software through the creation of a room dedicated to that patient. \textsc{ehe}s will be posted to this location by the various users and the information shown to them will be a reflection of the group consensus. These rooms contain a running ledger of posted EHEs and discussions about the patient necessary for users to discharge their roles.%

  \textsc{ehe}s posted to the room comprise the available details about a patient. Events posted by and visible to DuraChain users include:%



\tabulinesep=2mm
\begin{tabu} to \linewidth
  {|X[-2,l]|X[-2,c]|X[-2,c]|X[-2,c]|}
\cline{2-4}
\multicolumn{1}{c}{} & \multicolumn{3}{|c|}{\textbf{Electronic Healthcare Events}} \\ \cline{2-4} \hline
\rowcolor{gray!20}
Patient Name & \texttt{firstName} & \texttt{middleName} & \texttt{lastName} \\ \hline
Contact Information & \texttt{homePhone} & \texttt{mobilePhone} & \texttt{homeAddress} \\ \hline
\rowcolor{gray!20}
Primary Insurance & \texttt{primaryInsProvider} & \texttt{primaryInsContract} & \texttt{primaryInsGroup} \\ \hline
Secondary Insurance & \texttt{secondaryInsProvider} & \texttt{secondaryInsContract} & \texttt{secondaryInsGroup}\\ \hline
\rowcolor{gray!20}
Basic Info & \texttt{height} & \texttt{weight} & \texttt{gender} \\ \hline
\multicolumn{4}{|c|}{\textbf{Possible Events for Future Inclusion}} \\ \hline
\rowcolor{gray!20}
Diagnosis & \texttt{dsmCode} & \texttt{icdCode} & \\ \hline
Billing & \texttt{cptCode} & & \\ \hline
\end{tabu}


  \subsubsection{Users}
  The most critical aspect involved in ensuring robust security of data stored in patient rooms is the management of user permissions. Thus, our system only allows users to join a patient's room via an invitation issued by an existing, qualified member of that room. In the event that a patient is new to the system and does not yet have a room built for them, then a qualified, permissioned user associated with the provider making contact with that patient will be able to initiate the creation of a Patient Room.%

  \paragraph{Administration}
  Administrative users are the engineers and developers who work on the DuraChain project. While they will not have access to any Patient Room directly, they are responsible for the development and maintenance of the source code associated with DuraChain. An administrator has no access to nor knowledge of any room they are not invited to, further doubling down on security. However, they may be invited to any Patient Room in order to roll out updates, conduct maintenance, or debug issues that arise for our clients.%

  %\paragraph{Sales Representative}
  % Sales Representative is a certified Assistive Technology Professional (ATP) through Resna which allows them to complete the necessary forms for patients. These users will be in-charge of higher tier facilities in their area and be able to create a Patient Room when appropriate. These users will be able to post EHEs to a patient’s EHR ledger & perform many of the tasks available within the Patient Room.

  %\paragraph{Overlay Representative}
  %An Overlay Representative is a less experienced version of a Sales Representative. They will often take lower tier facilities in the area and works with less complicated DME. They will have a Sales Representative who will be working with them that will need to confirm any EHEs they may post to a a patient’s EHR ledger. As they progress and become ATP certified, these users end up becoming Sales Representatives.

\subsubsection{Facilities and Groups of Patients}
Goals for our software extend beyond a reimagination of how \textsc{ehr}s are handled and stored. We also aim to balance these novel data protocols with the most approachable and intuitive \textsc{uxui} in the industry. To this end, the software allows healthcare facilities to organize and group their patients in a way that allows for facility users to quickly sort through patients they care for.%

In order to promote easy navigation of the vast numbers of patient rooms that will be created, we employ the built-in Group function of the Matrix protocol. This allows patients to be organized according to the facilities that care for and service them.%

A user who is responsible for managing patients within a facility has the ability to create a new Patient Room---if the patient does not yet have one---and associate it with the facility's group(s). Grouping of patient rooms allows for an intuitive way for a facility to organize their patients and our \textsc{uxui} ensures a streamlined approach to this functionality.%

 \subsection{Implementation}
 Successful implementation of the DuraChain environment requires active adoption and participation by \textsc{dme} providers, who constitute our main client base.%

 In order to bring a client into our environment, they must have a copy of our server software properly installed and attached to their domain. From there, they can access our environment via our custom UI and begin creating Patient Rooms. To join a previously-existing Patient Room, they must be invited to an appropriately credentialed user inside of that room.%

\subsubsection{Protected Health Information}
Any discussion of healthcare data must obligatorily address the topic of Protected Health Information (\textsc{phi}). For DuraChain to be implemented properly, compliance with the Health Insurance Portability and Accountability Act of 1996 (\textsc{hipaa})\cite{HIPAA} is absolutely mandatory for both users and the servers housing patient data.%

By placing the server instances within the domain of the \textsc{dme} provider, the client assumes responsibility for creating users, Patient Rooms, and \textsc{hipaa} compliance more broadly. As the vendor of the DuraChain software, we assume liability only for performing due diligence on our clientele and extracting an assurance of their compliance with \textsc{hipaa}.%

In this way, the clients assume liability for compliance and removes that burden from us. This is not to a hedge or a dodge. In fact, it is actually the most efficient way to handle \textsc{hipaa} compliance and liability. Assuming that we only distribute our software to reputable and responsible \textsc{dme} businesses, who have a \textsc{hipaa} mandate, then our software remains compliant as long as all server instances are maintained by compliant parties.%
